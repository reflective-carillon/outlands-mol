\documentclass{article}

\begin{document}

Put image here: device


\begin{center}
{\Huge
Kingdom of the Outlands

Minister of the Lists
Handbook

}\end{center}



Revised 10/2005
Maria Abramsdottir
Kingdom MOL



\tableofcontents Contacting Kingdom Minister of the List

Contact information can be found in the Kingdom Officer list in the Outlandish Herald, by email at
mol@outlands.org, or on the Kingdom Minister of the List website through the Outlands.org
Officer’s webpage. The website contains online tournament and authorization reporting forms,
waivers, Kingdom Armored Combat Authorization List and the MOL Handbook.


\section{Responsibilities and duties of Ministers of the List}

It is vital to remember that the integrity and impartiality of the Lists is determined by the Minister of the
List. It is imperative that all MOLs remain above reproach.
You must treat all fighters fairly, withou regard to preferences or status.
If you make an exception for one fighter, you must make an exception
for all fighters. Maintaining a reputation of fairness, keeping accurate records, and following the rules
of the list will ensure a good tournament for you and all others involved.


Membership All SCA officers must be paid members of the SCA, Inc. for the duration of their office
(two years), and have access to The Outlandish Herald. Updates in policy and procedures
will be published in the Kingdom MOL submission to the Outlandish Herald.


Participation Local MOLs are expected to attend all locally sponsored events where tournaments will be
held, or appoint a suitable replacement (such as a deputy or MOL-at-Large). Local MOLs
are also required to attend local business meetings. Should an emergency arise and they

are unable to attend, a copy of the monthly report should be forwarded to the deputy or
seneschal to report to the populace in the MOL’s place.

List Rules All MOLs are required to enforce Kingdom and Society Laws and Policies related
to list procedures, and to ensure that any fighter entered into the list is eligible to compete.
Reporting Local MOL --- reports are sent to KMOL, local Seneschal, and Knight Marshal within 14
days of any local tournament. It is the local MOLs responsibility to report, even if a deputy
or MOL-at-Large conducted the list. Missing three consecutive reports is grounds fo
suspension or removal from office. (See appendix for sample form)






MOL-at-Large --- give reporting information to the local group MOL, or submit report to


KMOL and forward a copy to the local MOL for their report to other local officers.
Deputy All local MOLs must recruit and train an emergency deputy (drop dead) who will become
successor to the office. At the end of term, letters are submitted to the KMOL from both

the outgoing MOL and incoming MOL indicating the change of office, and new contact

information. The KMOL will then ensure the group’s acceptance of the incoming officer.


MOLs should be able to:
\begin{itemize}
\item remain organized, and deal efficiently with large amounts of paperwork,


\item form a working relationship with all warranted marshals within their local group,
\item maintain contact with KMOL, Knight Marshal and Seneschal and provide current contact

information including mailing address, phone number, and email address (if available),


\item effectively run all standard tournament forms (single and double elim, round robin, bear pit) and
be willing to learn formats with which they are unfamiliar,

\item assist the KMOL office in maintaining an accurate list of authorized fighters in the Kingdom by
participating in a census, as needed, of all authorized fighters in the local group. It is
recommended that a list of local fighters be maintained by the local MOL including SCA name,
membership number, membership expiration, authorized forms and authorization expiration.
\end{itemize}


\section{Laws, Policies and Customs of the Society and the Outlands}
All fighters must show proof of membership, waiver and authorization
before being allowed to engage in any combat activity.


Proof of membership can be established with:
\begin{itemize}
\item Membership card

\item Current Outlandish Herald or Tournaments Illuminated, addressed to the fighter

\item Postcard from Member Services, addressed to fighter

\item Website membership confirmation form

\item Kingdom Armored Combat Authorization List
\end{itemize}
A waiver must be signed if the fighter is using any form of membership other than a
signed blue membership card, indicating that a waiver is on file with SCA, Inc. If ever in doubt, always have the
fighter sign a waiver.

Authorization can be established with:

\begin{itemize}
\item A current authorization card (current authorizations from other Kingdoms are acceptable)

\item Kingdom Armored Combat Authorization List
\end{itemize}

Check that the fighter is authorized in the fighting style of the tournament (rattan or rapier). It is not the
responsibility of the MOL to check for particular forms (two handed weapon, fencing with defensive
secondary) unless asked to do so by the Marshal-in-Charge.


If there is any question about membership, waiver or authorization status, refer the fighter to the Marshal-
in-Charge or Knight Marshal. Under no circumstances may a fighter compete without complying with the
membership, waiver and authorization requirements. Any fighter attempting to fight without complying
or who refuses to comply should be reported to the Marshal-in-Charge immediately.

It is Outlands tradition to use only one title in the list, usually the highest rank received. However, it is at
the fighter’s discretion to use the title they prefer. For example, a Countess who has received a White
Scarf may prefer to enter the list with the title of Dona instead of Countess or Her Excellency.

\section{Advance Preparation}
\subsection{Communication}

Once the list is open, the MOL needs to concentrate on the fighters and the tournament. Try to prepare as
early as possible to save frustration and confusion at the list field. Schedules, tournament styles and extra
help can be arranged far in advance. Contacting the appropriate persons before the tournament will make
each tournament run much smoother.

Autocrat Find out schedules (list opening and closing as well as tournament
time), accommodations provided (shade, table, chairs, etc.), location
of list field, time and place to check in on site, names and contact
information for the tournament sponsor (if applicable), Marshal-in-
Charge, Head Field Herald and the Chirurgeon.

Tournament sponsor Find out tournament styles, and if not decided, what is most likely to
be run (for example, under 10 fighters, round robin; 10-20 fighters, 2
of 3 double elim; over 20 fighters, speed tourney) how the bouts
should be drawn (random, expedited, tree) and how bye fights will
be handled (destructive or nondestructive). Most sponsors will have
some idea what they want to do.

Additional MOLs If another MOL will be conducting the tournament, ensure they will
be able to attend, that they have adequate supplies, explain the
tournament information, and contact information for the tournament
report.

Marshal-in-Charge Find out when armor inspections will be conducted, the need for
additional list fields, the need for bye fighters, and any additional
concerns or questions.

Field Herald Let the herald know when announcements need to be made, how the
cards should be handled, and ask if there is anything the MOL can
do to help the herald(s), for example, phonetic spelling on the list
cards.

Card runners Try to set up reliable card runners, if necessary, ahead of time. This
is a great job for new or young people who are planning to be at the
list field anyway. Arranging in advance allows adequate time to
explain how to handle the cards and any other expectations. Talk to
parents for permission before asking children to run cards. Check
with runners directly prior to tournament to make sure they are still
able to run.

Other obligations Contact anyone to whom you have obligations (fighters, family, Peer
or Nobility) and advise them as to when you will and will not be
available to them.

\subsection{Supplies and Setup}
Plan to take adequate supplies for all tournaments you are conducting. It is advisable to take double the
amount anticipated in list forms, waivers and list cards. This will make certain you have prepared for any
unforeseen circumstances.

Also, remember to bring items to ensure your comfort during the tournament. You may be at the list far
longer than expected. Weather can change unpredictably. You will enjoy yourself far more if you are
amply prepared for all possibilities.

Do the best you can to remain self sufficient. If something is not prepared, try to see to it yourself. The
autocrat(s) and/or tournament sponsor will usually be busy at tournament time, and will greatly appreciate
having one less thing to worry about.

List Supplies
\begin{itemize}
\item List forms
\item List cards
\item Waivers (adult and minor)
\item Armored Combat Authorization List
\item Tournament reporting forms
\item Pencils, pens, highlighters or markers, erasers
\item Round robin pair sheets and/or tournament trees
\item Paperweights / Clipboard / paperclips
\item Duct tape and electrical tape
\end{itemize}

Comfort Items
\begin{itemize}
\item Chair or cushion
\item Snacks
\item Drinks
\item Sunscreen
\item Bug repellant
\item Lip balm
\item Blanket
\item Umbrella / parasol
\item Cloak / coat / gloves / hat
\item Book or handwork
\end{itemize}

\section{Listing a Tournament}
\subsection{The Basics}
This handbook is not meant to convey the ONLY way to run a list. There are several alternative ways that
may work better for you. Look for the way that works best for you and use it.
\subsubsection{Summary of Activities}

1. Upon your arrival check in with the Autocrat, Marshall in Charge and Herald to coordinate
schedules and information. You should arrive early enough to have time to seek everyone out and to
set up the equipment you will need.

2. Check the location of the MoL spot. If it is too far away from the fighting field, a direct path
between you and the field may be impossible, making it hard for the runners to return the cards to
you. You should attempt to set up in a location where you have an unobstructed view of the field,
and that is convenient for the runner to return cards to you.

3. Herald opening of list. Prepare card runners. List the tournament

4. Advise the tournament sponsor, Autocrat, and/or Local or Ruling Nobility of the winner(s) of the
tournament upon completion.

5. Pack your belongings, clean up your area, and enjoy the rest of the event.

6. Send in your report to the Kingdom Minister of the List, local seneschal and Knight Marshal.



Put image here: position near the field



\subsubsection{Completing the list form}

The List Form is a piece of numbered paper upon which the fighters’ name (including title) is listed. You
will use the number to identify this fighter for the rest of this tournament. This is just a time saving
method for you. See the Appendix for list forms.

1. Once the Herald has announced the list is open, if not before, fighters will begin showing up. Make
sure you know the type of tournament being fought because the fighters are sure to ask you.

2. Ask the fighter for their Membership and Authorization Cards. If there are multiple tournaments
scheduled for one day, or event, you need only see the membership and authorization cards once.

3. Use these cards to record the fighter’s name on the list paper. Ask the fighter for the title that they
wish to use if they have multiple titles, or if they have a title if you are unsure. Please note that the title
"Squire", "Cadet" or "M’Lord / Lady" are not acceptable titles for a tournament.
Sample List Form:

Name of Tournament: Date:

Fighter Name 1 2 3 4 5 6 7 8
1 Sir Winnie the Pooh

2 Lord Owl
3
The Honorable Lord (THL)
Christopher Robin

4

Put image here: list form


4. Across each row are blocks. Half of the block will be filled with the fighter’s number that person is
paired to fight in that specific round and half will eventually contain a "W" for win, "L" for loss or
"B" for a nondestructive bye.


\subsubsection{Completing the list cards}

Prepare the list cards while there is a break in the fighters entering the list.
1) The list cards are what the Heralds will use to call the bouts, please make sure these cards are legible.
2) Each fighter in the tournament should have a list card. Creating a list card for the bye fighter is
optional, it is up to you.

Sample list card:

put image here: list card

\#\# = Fighter number from list form
Fighter name, with title
R\# = round number
Number 2 (\#\#)
Fighter name
Lord Owl
VS W/L
R1
R2
R3
R4


\subsection{Pairing Fighters into Bouts}
Once the list is closed, pair up each of the fighters. This can be done in several ways, depending on the
tournament style.

Round Robin

If there are more than 12 fighters entering a round robin, split the fighters into two fields to decrease the
overall numbers. Too many fighters will mean that the fighting will continue long into the night, since
every fighter must fight every other fighter in the tournament.

Using predetermined pairing list (see appendix) is the easiest method for drawing a round robin. All that
you will need to do is assemble the cards according to the preprinted pairings and record the results. If a
pairing list is not available, another easy method of pairing is to rotate through the list. (example: fighter
\#5, R1 vs. \#6, R2 vs. \#1, R3 vs. \#2) No fighter will be eliminated because each fighter will fight every
other fighter on the field, no matter the number of wins or losses.

If a fighter has to drop out of a Round Robin Tournament before the completion of the tournament, you
have a couple of options:
\begin{itemize}
\item You can assign byes to each fighter that still had to fight this person. (Best with an even number)

\item You can redraw the list and repair all of the fighters, ensuring that each of the fighters fights everyone
else. (Best with an odd number)

\item Another option, although rarely used, is to allow a fighter to join in the tournament in the place of the
fighter who had to withdraw. This person must accept the number of wins and losses of the fighter that
they are replacing. This option is up to the discretion of you and the tournament sponsor.
Single and Double Eliminations
\end{itemize}
The first rounds of both single and double elimination tournaments are paired by the MOL or tournament
sponsor, using a random draw method. Use the method you or the tournament sponsor is most
comfortable with.

When appropriate, present the cards to the tournament sponsor for their review prior to completing noting
each fight. This will allow a review of the pairings and a chance to change the pairings. If a squire and
their knight, or two fighters who always fight each other are paired, it is appropriate to reshuffle or redraw
that bout, with the tournament sponsor’s permission.

After the first round, the tournament style will dictate how to continue pairing. If possible, begin drawing
the next round while the current round is being fought, this will speed up the tournament run time. If you
or the tournament sponsor prefer, you can wait until the end of the round to draw the pairing for the next
round. Do not allow the tournament sponsor, fighters, or MIC to rush the pairing process. They usually
want to get onto the next round’s fighting, and can become irritated by delays. Mistakes are usually made
when the MOL is in a hurry.


Single Elimination

Once the fighter receives one loss, they are no longer able to fight in the tournament. Their List Card is
removed from the pile and placed to the side. An indication is made on the List Form; this is usually a line
that is drawn through the remainder of the rounds for that fighter. The pool of fighters will continue to
decrease as fighters with a loss are eliminated.

Double Elimination

The first two rounds are almost always random draws. Once the fighter receives two losses, they are no
longer able to fight in the tournament. The MOL must make certain that paired fighters have not fought
each other. When a fighter has been paired with all other fighters remaining in the list, only then can
fighters be paired again. After the first two rounds, an expedited pairing method can be used.

Random Draw Methods
\begin{itemize}
\item Take the top card in the stack and the bottom card in the stack. This ensures that people who signed up
together do not necessarily fight each other. Continue taking the top and bottom card until all fighters
have been paired. If you have one card left over, assign that person the bye.

\item Take the stack of cards and shuffle them thoroughly. Take the top two cards; this is your first pair.
Continue taking two cards until all fighters have been paired together. If you have one card left over,
assign that person the bye.

\item Randomly, draw two cards and pair them up. Continue randomly drawing two cards at a time until all
fighters are paired up. If you have one card left over, assign that person the bye.

\item Place cards in two equal piles, using a predetermined method of splitting the list (example: chivalry vs.
non-chivalry, local vs. out-of-area) Shuffle each stack and pair fighters from opposite stacks. If you
have one card left over, assign that person the bye.

\item Use tokens, or pieces of paper with fighter name or number written on it. Draw tokens from cup or
pouch, pairing fighters as drawn.

\item Use of a tournament tree.
\end{itemize}

Expedited Draw

A method of speeding up a double elimination tournament can be made in the pairing process. Talk to the
tournament sponsor to find out what draw method to use. Unless specifically asked to use an expedited
draw, use the random draw method throughout the tournament.
Place list cards in two stacks, one for fighters who have no losses, one for fighters who have one loss.
Randomly draw from each pile, pairing fighters with one loss against each other and fighters without a
loss against each other. This guarantees fighters will be eliminated in each round, usually making the
tournament slightly shorter.


\subsection{Recordkeeping}
Recording Bouts

Once all of the fighters are paired, write the bouts on each card. This will act as a double check to ensure
that fighters do not fight each other multiple times in the tournament, unless necessary. (Many MoL’s will
record the bout order on the cards, along with the fighters. Adding an additional number or letter
indicating which pair is first, second, etc.) An example:

Put image here: card

Number 1 Number 2
Fighter name Fighter name
Sir
Winnie The Pooh Lord Owl
VS W/L VS W/L
R1 2 (A) R1 1 (A)
R2 R2
R3 R3
Then, transfer the bout information to the list form. An example:

Put image here: list form

Name of Tournament: Date:
Fighter Name 1 2 3 4 5 6 7 8
1 Sir Winnie the Pooh 2
2 Lord Owl 1
3
The Honorable Lord (THL)
Christopher Robin Bye
4
Once all of the information has been checked and recorded, look at each pair and ensure that they are
stacked in the order of precedent; the person with the most precedence is the first of the two in the pairing.
If you have any questions, ask the Local or Kingdom Herald present. Once you have reviewed the
pairings, give the cards to the Herald, and prepare to record the wins and losses of each fight. (If multiple
fields are being used, prior to giving the cards to the Herald, divide the pairings into the appropriate
number for each field.)

The Herald will then call all fighters to the field and announce the fighters in each bout.
Recording Bout Results

Once each bout is completed, the list cards should be returned to you with the winner on top. Because
there are sometimes questions on the field, a fight is re-fought, or other conditions arise, you should wait
to record the results of the bout until the cards have been returned to you. This practice may save you
some confusion as you draw later rounds. If there is any question regarding the outcome of the fight,
PLEASE ask the Marshall, Herald, or fighters.

Next to the appropriate round, record "W" on the winning fighter’s card, "L" on the losing fighter’s card
and transfer to the List Form.

Continue with this process until the tournament is completed.

\section{Commonly Used Terms}

\begin{itemize}
\item Bout (Match) --- two fighters or teams competing until one is defeated.
\item Bye --- when there are an odd number of fighters in a round, one fighter will not have an opponent, and
a bye is given. The tournament sponsor or MIC will determine how byes will be handled.
\item Bye fighter --- in some cases, a designated bye fighter will be paired against the fighter drawing
the bye. Otherwise, the fighter drawing the bye will not fight in the round.
\item Non-destructive bye --- the bout does not result in a win or loss for the fighter, regardless of the
outcome. Record just the bye on the list form and card. Non-destructive byes can provide an
advantage in a tournament; take care to only give one bye fight to any given fighter.
\item Destructive bye --- the results of the bye fight are recorded as a regular win or loss.
\item List --- the names of all fighters participating in a tournament.
\item List cards --- cards (or pieces of paper) given to the Herald to call each bout. These cards can also be
used to pair fighters, and to record wins and losses.
\item List form --- form (or sheet of paper) used to track all the fighters names, all bouts in each round, and
wins and losses for each bout.
\item Marshal-in-Charge (MIC) --- the person in overall control of the fighting, and the tournament.
\item Minister of the List (MOL) --- the person running the list, or paperwork, aspect of a tournament. The
Mol pairs fighters in each bout, tracks wins and losses, and any other requested information.
\item Tournament sponsor --- the person (or people) deciding the form and conditions of the tournament.
The sponsor is often the previous winner, ruling Nobility or the MIC.
\end{itemize}

\section{Tournament Types}

\begin{itemize}
\item 2 out of 2 (2/3) ---The winner of a bout is determined by winning two of three fights. Unless requested
otherwise, the MOL records the overall winner, not each individual fighter’s wins and losses per bout.

\item Bear Pit --- Within a designated space, the initial fighter holds the field (fighting area). Challengers
line up to fight for control of the field. The winner remains on the field to challenge the next fighter,
and the losing fighter goes to the back of the line, to challenge the field again. The Herald or fighter
announces the winner to the MOL. The MoL tracks only the number of each fighter’s wins, or the
length of time each fighter holds the field.

\item Speed or Lightening --- a bear pit format with a specified time limit.
\item Cumulative Wounds --- Any wounds that a fighter receives during a fight are carried over to the next
round. This means that is a fighter loses a limb, but wins the bout, in the next round that limb is still
lost to the fighter.

\item Double Kill --- Both fighters kill each other simultaneously. It is up to the tournament sponsor whether
the fight will be re-fought, or if both fighters receive a loss.

\item Elimination --- Fighters are dropped from the tournament through losses.
\begin{itemize}
\item Single --- (single elim) When a fighter has lost one bout, the fighter is eliminated.

\item Double --- (double elim) When a fighter has lost two bouts, the fighter is eliminated.
\end{itemize}
\item First Blood - This type of tournament is typically done for the light weapons fighters. The fight
continues until the "blood" is drawn, (I.e., a blow is received in a non-fatal area such as an arm or leg).
The fighter receives a loss when the wound is fatal.

\item Line tournament --- (variation of a round robin) Fighters form two lines, facing each other. The
fighters directly across from each other are opponents. Once the fights are over, the MOL record wins
and losses from each line.

\item Melee --- Battle scenarios fought by teams. Although the MOL usually does not need to track anything
in a melee, cards do need to be checked before any fighting occurs.

\item Restricted Weapons --- A tournament limited to a certain weapons form. The MOL needs to check the
authorization card to ensure that the fighter is currently authorized in that weapons form.

\item Round Robin --- Each fighter in the tournament will fight all other fighters. Fighters are paired into
bouts, and all wins and losses are recorded. The winner is the fighter with the most wins overall. See
the appendix for pairing lists.

\item Snowball --- The first round is drawn and the fighters are paired off. Whoever wins the fight becomes
the team leader and the losing fighter joins the winning fighter’s team, forming a two-man team. In the
second round, the two-man teams are paired off, with the losing team joining the winning team,
forming four-man teams, etc. This continues until there are only two teams left. The two teams then
fight to determine the winner. This tournament works best with an even number of fighters.
\end{itemize}

Put image here: list form with slashes and numbers

Put image here: list form with slashes


Put image here: list form without slashes and with numbers

Put image here: list form without slashes

Put image here: card

Put image here: report

Put image here: round robin



\end{document}
